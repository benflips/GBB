\documentclass[]{article}
\usepackage{lmodern}
\usepackage{amssymb,amsmath}
\usepackage{ifxetex,ifluatex}
\usepackage{fixltx2e} % provides \textsubscript
\ifnum 0\ifxetex 1\fi\ifluatex 1\fi=0 % if pdftex
  \usepackage[T1]{fontenc}
  \usepackage[utf8]{inputenc}
\else % if luatex or xelatex
  \ifxetex
    \usepackage{mathspec}
    \usepackage{xltxtra,xunicode}
  \else
    \usepackage{fontspec}
  \fi
  \defaultfontfeatures{Mapping=tex-text,Scale=MatchLowercase}
  \newcommand{\euro}{€}
\fi
% use upquote if available, for straight quotes in verbatim environments
\IfFileExists{upquote.sty}{\usepackage{upquote}}{}
% use microtype if available
\IfFileExists{microtype.sty}{%
\usepackage{microtype}
\UseMicrotypeSet[protrusion]{basicmath} % disable protrusion for tt fonts
}{}
\usepackage[margin=1in]{geometry}
\ifxetex
  \usepackage[setpagesize=false, % page size defined by xetex
              unicode=false, % unicode breaks when used with xetex
              xetex]{hyperref}
\else
  \usepackage[unicode=true]{hyperref}
\fi
\hypersetup{breaklinks=true,
            bookmarks=true,
            pdfauthor={Ben Phillips, phillipsb@unimelb.edu.au},
            pdftitle={Genetic backburn},
            colorlinks=true,
            citecolor=blue,
            urlcolor=blue,
            linkcolor=magenta,
            pdfborder={0 0 0}}
\urlstyle{same}  % don't use monospace font for urls
\setlength{\parindent}{0pt}
\setlength{\parskip}{6pt plus 2pt minus 1pt}
\setlength{\emergencystretch}{3em}  % prevent overfull lines
\setcounter{secnumdepth}{5}

%%% Use protect on footnotes to avoid problems with footnotes in titles
\let\rmarkdownfootnote\footnote%
\def\footnote{\protect\rmarkdownfootnote}

%%% Change title format to be more compact
\usepackage{titling}
\setlength{\droptitle}{-2em}
  \title{Genetic backburn}
  \pretitle{\vspace{\droptitle}\centering\huge}
  \posttitle{\par}
  \author{Ben Phillips,
\href{mailto:phillipsb@unimelb.edu.au}{\nolinkurl{phillipsb@unimelb.edu.au}}}
  \preauthor{\centering\large\emph}
  \postauthor{\par}
  \predate{\centering\large\emph}
  \postdate{\par}
  \date{16/03/2015}




\begin{document}

\maketitle


\section{Abstract}\label{abstract}

\emph{Stuff}

\section{Introduction}\label{introduction}

More stuff.

\section{Methods}\label{methods}

To investigate this idea, we built a spatially-explicit simulation model
that tracks a population spreading through space. The population is
composed of sexually hermaphroditic individuals with discrete
generations.

\subsection{Population dynamics}\label{population-dynamics}

Each individual in the population has a maximum density-independent rate
of reproduction, \(R_i\), governed by a universal maximum possible
fecundity, \(R_{max}\), and an individual fitness modifier, \(w_{i}\),
such that \(R_i=w_{i}R_{max}\). An individual's expected reproductive
output, \(\text{E}(W_i)\) is further modified by density dependence,
described using the Beverton-Holt model:

\[ \text{E}(W_i)=\frac{R_i}{1+\alpha N}\]

where \(\alpha=\frac{R_{max}-1}{N^*}\). In this formulation \(N^*\)
represents the carrying capacity that would be achieved if all
individuals achieved a fecundity of \(R_{max}\).

Finally, we introduced demographic stochasticity into the model by
drawing each individual's expected fecundity from a poisson
distribution: \(W_i\sim\text{Poiss}(\text{E}(W_i))\).

\subsection{Spatial dynamics}\label{spatial-dynamics}

We set this population up on a 1D lattice of contiguous cells, with
local dynamics playing out within each cell on the lattice. Individuals
disperse in continuous space, but are aggregated to the cell for the
purposes of local dynamics. Immediately after birth (and the death of
parents), individuals disperse according to their individual dispersal
phenotype, \(d_i\). Dispersal is treated as a stochastic process with
each individual displacing according to a draw from a normal
distribution with a mean equal to its current location in space, \(x_i\)
and a standard deviation equal to \(e^{d_i}\). Offspring inherit their
parent's location, and so commence dispersal from that point (rather
than the centre of the cell).

\subsection{Evolutionary dynamics}\label{evolutionary-dynamics}

Each individual carries with it 20 independently-segregating diploid
loci. There are two alleles possible at each locus: one that is neutral
with regard to dispersal, and the other which increases the
\(\text{E}(d_i)\) of the individual by an effect size, log(0.1), that is
constant across all loci. The total additive effect of all alleles
within an individual generate that individual's \(E{d_i}\). The
individual's realised dispersal phenotype is, however, subject to a
random environmental effect, such that
\(d_i=\text{N}(\text{E}(d_i), \sqrt{V_e})\), where \(V_e\) is the
environmental variance. This environmental variance is constant across
space, time, and individuals and is determined at the initialisation of
the simulation such that the initial heritability of \(d_i\) is
approximately 0.3.

\subsection{Plots}\label{plots}

\subsection{Citations}\label{citations}

The relationship was first described by Halpern et al. (2006). However,
there are also opinions that the relationship is spurious (Keil \emph{et
al.} 2012). We used R for our calculations (R Core Team 2014), and we
used package \texttt{knitcitations} (Boettiger 2015) to make the
bibliography.

\section{Results and discussion}\label{results-and-discussion}

Lorem ipsum dolor sit amet, est ad doctus eligendi scriptorem. Mel erat
falli ut. Feugiat legendos adipisci vix at, usu at laoreet argumentum
suscipiantur. An eos adhuc aliquip scriptorem, te adhuc dolor
liberavisse sea. Ponderum vivendum te nec, id agam brute disputando mei.

Putant numquam tacimates at eum. Aliquip torquatos ex vis, mei et quando
debitis appareat, impetus accumsan corrumpit in usu. Nam mucius facilis
singulis id, duo ei autem imperdiet instructior. Cu ceteros alienum mel,
id vix putant impedit, ex idque eruditi forensibus eum. Posse dicunt id
usu. Ei iracundia constituto sed, duo ne exerci ignota, an eum unum
conceptam.

Has audire salutandi no, ut eam dicat libris dicunt. Pri hendrerit
quaerendum adversarium ea, dicat atqui munere et sea. Illum insolens eos
ne, eu enim graece rationibus mea. At postea utamur mel, eius nonumes
percipitur at vis. Numquam similique in per, te quo saepe utroque
pericula.

Ea nonumy volumus usu, no mel inermis dissentias. Dico partiendo
vituperatoribus eum et. Mea accusam convenire te, usu populo qualisque
gloriatur ut. Eu eum oratio altera option, ad mea ignota scriptorem. Ne
suas latine vix, eos oblique sanctus pertinax cu.

\section*{References}\label{references}
\addcontentsline{toc}{section}{References}

Boettiger, C. (2015). \emph{Knitcitations: Citations for knitr markdown
files}. Retrieved from \url{https://github.com/cboettig/knitcitations}

Halpern, B.S., Regan, H.M., Possingham, H.P. \& McCarthy, M.A. (2006).
Accounting for uncertainty in marine reserve design. \emph{Ecol
Letters}, \textbf{9}, 2--11. Retrieved from
\url{http://dx.doi.org/10.1111/j.1461-0248.2005.00827.x}

Keil, P., Belmaker, J., Wilson, A.M., Unitt, P. \& Jetz, W. (2012).
Downscaling of species distribution models: A hierarchical approach (R.
Freckleton, Ed.). \emph{Methods Ecol Evol}, \textbf{4}, 82--94.
Retrieved from \url{http://dx.doi.org/10.1111/j.2041-210x.2012.00264.x}

R Core Team. (2014). \emph{R: A language and environment for statistical
computing}. R Foundation for Statistical Computing, Vienna, Austria.
Retrieved from \url{http://www.R-project.org/}

\end{document}
